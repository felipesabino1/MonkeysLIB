\chapter{Algebra}
    
    \section{Inteiro Modular}
    Classe para lidar com problemas relacionados a operações modulares.
    
    \lstinputlisting{./solutions/Algebra/modint.cpp}
    \section{Matrix Power}
    \subsection{Graph Paths I}
    Consider a directed graph that has $n$ nodes and $m$ edges. Your task is to count the number of paths from node $1$ to node $n$ with exactly $k$ edges.
    \lstinputlisting{./solutions/Algebra/GraphPathsI.cpp}
    
    \subsection{Graph Paths II}
    Consider a directed weighted graph having $n$ nodes and $m$ edges. Your task is to calculate the minimum path length from node $1$ to node $n$ with exactly $k$ edges.
    \lstinputlisting{./solutions/Algebra/GraphPathsII.cpp}
    \section{FFT}
    \subsection{Simple FFT}
    Multiplicação de dois polinômios em tempo $O((n+m) \cdot \log(n+m))$
    O polinômio $ a + bx + cx^2 + dx^3$ é representado na forma de vetor como $[a, b, c, d]$.
    Para usar a função $multiply$, crie dois vectors $vector<int> a, b;$ apos fazer $multiply(a, b);$ tem-se o vector polinômio com os coeficientes da multiplicação de a e b.
    \lstinputlisting{./solutions/Algebra/fft.cpp}
    \subsection{FFT To solve String with Wildcards}
    Given two strings $s$ and $t$, with $t$ having wildcards that can match with any character, returns the number of ocurrence of $t$ in $s$.
    Solution $O(nlogn)$.
    \bigskip
    (Original Problem: Maraton Nacional de Programacion Colombia 2016 - Wildcards).
    \lstinputlisting{./solutions/Algebra/fft_wild_card.cpp}
    \subsection{FFT modular (NTT)}
    To use a modular FFT, also known as NTT, you need a $n$-th root of ur modulus and then define some values in general u can say :
    
    $mod = x*2^k$
    
    $G = 3 $ or $5$
    
    $root = G^{mod/(2^k)}$
    
    $root_1 = root^{-1}$
    
    $root_{pw} = 2^k$
    \lstinputlisting{./solutions/Algebra/ntt.cpp}

    \subsection{FWHT (fft using xor/and/or)}
    FFT but the expoent operation is xor/and/or.

    There are some crazy problems where u need to multiply $N$ polynomials of the form $x^0 + x^i$ and the $i$ can be up to $N$. the main approach is to use some D\&C to multiply those and to keep the polynomials divide in two parts. It is also coded down here. Operations can be modular.
    \lstinputlisting{./solutions/Algebra/FWHT.cpp}
    \section{Gauss}
    \subsection{Markov Chains}
    if we have $E(i) = 1 + \sum_{j=0}^{N-1} P[i][j] * E(j)$

    We can model as a System of linear equations of the form:

    $E(i) - \sum_{j=0}^{N-1} P[i][j] * E(j) = 1$

    So:

    \[ \begin{bmatrix}
         1-P[0][0] & -P[0][1] & \cdots & -P[0][N-1]\\
         -P[1][0] & 1-P[1][1] & \cdots & -P[1][N-1]\\
         \vdots & \vdots & \ddots & \vdots\\ 
         -P[N-1][0] & -P[N-1][1] & \cdots & 1-P[N-1][N-1]\\
     \end{bmatrix}
     \times
     \begin{bmatrix}
         E(0)\\
         E(1)\\ 
         \vdots\\ 
         E(N-1) 
     \end{bmatrix}
      =
     \begin{bmatrix}
         1\\
         1\\ 
         \vdots\\ 
         1 
     \end{bmatrix} \]


    Works with probabilities as well, but without the 1 in the sum.

    Gauss Solves a System of linear equations in $O(min(N,M)*N*M)$, with $N$ being the number of variables and $M$ being the number of equations.

    Given a Directed Graph, find the probability of a random path goes from $1$ to $300$ and does not step in $290 \cdots 299$
    \lstinputlisting{./solutions/Algebra/gauss.cpp}

    
    \subsection{Modular}
    Given a tree (undirected unweighted graph), assign for each node in this a color $c_i$ ($0 \leq c_i \leq k - 1)$, such that for each node $i$, $c_i$ is the sum of colors adjacents to $i$ mod $k$.
    \bigskip
    Compute the number of differents possible colorings.
    \bigskip
    Solution $O(n^3)$.
    \bigskip
    (Original Problem: LightOJ 1279 - Graph Coloring).
    \lstinputlisting{./solutions/Algebra/GaussModular.cpp}
    \subsection{Modular 2 (bitset)}
    Problem:
    
    a grid of switchs and each press in a switch i,j swaps the state of himself and the ones which are adjacent to it.
    
    If there is a way to make all on, print the switchs pressed.
    \lstinputlisting{./solutions/Algebra/gaussmod2.cpp}
    
    \subsection{Matrix rank}
    The rank of a matrix is the largest number of linearly independent rows/columns of the matrix. The rank is not only defined for square matrices.
    We can see that the other variables are "free", so can be uses to combinatorial problems.
    
    Example:
    
    Given a set of n integers, how many subsets have xor not equal to 0?
    \lstinputlisting{./solutions/Algebra/subset-nim.cpp}

    \subsection{Xor Basis}
    Cebolinha me mostrou essa parada, fazendo uma solução pro mesmo problema acima usando isso.

    Da pra gente saber se um X pertence ao XOR de algum subconjunto dos seus caras.

    o total de caras q pertence ao subconjunto é $2^{B.size()}$.

    \lstinputlisting{./solutions/Algebra/xorbasis.cpp}
    
    \section{Chinese Remainder Theorem}

    % \subsection{Coprimes congruences}
    % Given a list $A$, and a list $M$, find a $X$ that $X = A_i mod M_i$. $X$ is given $mod LCM(M_1,\cdots, M_{n-1})$.
    % \lstinputlisting{./solutions/Algebra/crt.cpp}

    \subsection{Algorithm}

    Given $(a, n, b, m)$, find a $X$ such that $X = a \mod n$ and $X = b \mod m$. $X$ is given mod LCM$(n, m)$.
    \lstinputlisting{./solutions/Algebra/crt2.cpp}

    \subsection{Lucas Theorem}

    Given $n, k$ and $m$, find $\binom{n}{k}$ mod $m$, where $(n, k \le 10^{9})$, $m$ is a square-free number less than or equal to $10^{9}$ and each prime that appears on $m$ is less than $50$. A Square-Free number is a number that, in his prime factorization, each prime appears exactly once.

    \lstinputlisting{./solutions/Algebra/Lucas.cpp}
    
    \section{Simplex}
    Maximize ${\textstyle \mathbf {c^{T}} \mathbf {x} }$
subject to ${\displaystyle A\mathbf {x} \leq \mathbf {b} }$ and ${\displaystyle \mathbf {x} \geq 0}$
    \lstinputlisting{./solutions/Algebra/simplex.cpp}
    \section{GCD Array Trick}
    Way to store all gcds in a array range . For each $i$ you store every different gcd sub-segment that exists starting at $i$.
    
    In this problem you need to find a $Max(gcd(A[l],\cdots,A[r])*(r-l+1))$.
    \lstinputlisting{./solutions/Algebra/gcdrobado.cpp}
    \section{Implementation of Polynomials}
    Algorithm to find a root, and also to divide a poly by some (x+c).
    \lstinputlisting{./solutions/Algebra/polynomials.cpp}
    \section{Combinatorial}
    Multiple combinatorial solutions.
    
    Number of ways to take $K$ objects, between $N -> Comb(N,K)$
    
    Number of ways to put $N$ objects in $K$ boxes (stars and bars) $-> sb(N,K)$

    Count how to label $N + K$ pairs of parentheses with $K$ $'('$s already fixed $-> catalan(N,K)$
    
    \lstinputlisting{./solutions/Algebra/combinatorial.cpp}
    Version that Sabino likes
    \lstinputlisting{./solutions/Algebra/comb.cpp}
    \subsection{Derangements}
    The number of derangements of $n$ numbers, expressed as $!n$, is the number of
    permutations such that no element appears in its original position. Informally,
    it is the number of ways $n$ hats can be returned to $n$ people such that no
    person recieves their own hat.
    
    Principle of Inclusion-Exclusion
    
    Suppose we had events $E_1, E_2, \dots, E_n$, where event $E_i$ corresponds to
    person $i$ recieving their own hat. We would like to calculate $n! - \lvert E_1 \cup E_2 \cup \dots \cup E_n \rvert$.
    
    We subtract from $n!$ the number of ways for each event to occur; that is,
    consider the quantity $n! - \lvert E_1 \rvert - \lvert E_2 \rvert - \dots - \lvert E_n \rvert$. This undercounts, as we are subtracting cases where more
    than one event occurs too many times. Specifically, for a permutation where at
    least two events occur, we undercount by one. Thus, add back the number of ways
    for two events to occur. We can continue this process for every size of subsets
    of indices. The expression is now of the form:
    
    $$n! - \lvert E_1 \cup E_2 \cup \dots \cup E_n \rvert = \sum_{k = 1}^n (-1)^k \cdot (\text{number of permutations with $k$ fixed points})$$
    
    For a set size of $k$, the number of permutations with at least $k$ indices can
    be computed by choosing a set of size $k$ that are fixed, and permuting the
    other indices. In mathematical terms:
    
    $${n \choose k}(n-k)! = \frac{n!}{k!(n-k)!}(n-k)! = \frac{n!}{k!}$$
    
    Thus, the problem now becomes computing
    
    $$n!\sum_{k=0}^n\frac{(-1)^k}{k!}$$
    
    which can be done in linear time.
    \section{Lagrange Interpolation}
    Suppose u have a polynomial of order $K$, and want to find $f(N)$, and N is really big.

    With Lagrange Interpolation, we can define any $f(x)$ as a linear combination of k+1 $f(u)$'s.

    The normal approach is to calc the first k+1 answers and use the to calculate $f(N)$.

    Math below:

    given the set of points .

    $l_j(x) = \frac{x-x_0}{x_j-x_0} * \cdots * \frac{x-x_{j-1}}{x_j-x_{j-1}} * \frac{x-x_{j+1}}{x_j-x_{j+1}} * \cdots * \frac{x-x_k}{x_j-x_k} $

    $f(x) = f(x_0)*l_0(x) + f(x_1)*l_1(x) + ... f(x_{k-1})*l_{k-1}(x) + f(x_k)*l_k(x) $
    
    \lstinputlisting{./solutions/Algebra/lagrange.cpp}
    Felipe Sabino, aka sabino1, version.
    \lstinputlisting{./solutions/Algebra/lagrangefelipe.cpp}
    % Teoria dos numeros vai ter uma seção propria
    \section{Discrete Log}

    \tab You have an modular equation with $a,b,x \in \mathbb{Z}$, you want to find a value for $x$
    
    $a^x \equiv b \;(\bmod\; m)$

    If you only want to find a solution, you just have to consider values between $[0,m-1]$ because, by the Pigeonhole Principle, $a^0,a^1,\dots,a^m$ you can only have m possible values, so minimally $a^m$ it's going to be a value that already appeared and it's going to cycle. So you have to consider values between $[0,m-1]$.

    However, the values of $m$ can be big, $1e9$, there's a way to solve it in $s\sqrt{m}$, the Baby-step giant-step.

    Let's consider that $gcd(a,m) = 1$ and rewrite the expression (remember you can write every integer with $x = q*p + r$, where $0 \leq r < p$, and with that you can also write $x = (q+1)*p -r$, where $0 \leq r < p$):

    $$
    a^x \equiv b \mod{m} 
    $$
    $$
    a^{q*p-r} \equiv b \mod{m}
    $$
    $$
    a^{q*p}*a^{-r} \equiv b \mod{m}
    $$
    $$
    a^{q*p} \equiv b*a^r \mod{m}
    $$

    If you determine a good value for $p$, you can decrease the range of possible values of both $q$ and $r$, let $p = \sqrt{m}$, then $0 \leq p \leq \sqrt{m}+1$ (the plus 1 comes from the minus from the $r$) and $0 \leq r < \sqrt{m}$, so you can precalculate the values of $a^{q*p}$ for all $0 \leq p \leq \sqrt{m}$ and $b*a^{r}$ for all $0 \leq r < \sqrt{m}$, so in the end you have $O(\sqrt{m})$. And to find values of $q$ and $r$, you go through the values of $a^{q*p}$ and use binary search or two point to find a value $b*a^r$ that is equal.

    If $gcd(a,m) \neq 1$, then you can't calculate the inverse of $a^r$, however you can still try to solve it, let $g = gcd(a,m)$, if $g \nmid b$ then the equation doesn't have solution (if $g \mid a$ and $g \mid m$, then certainly, by linear combination, $g \mid b$). If $g \mid b$, then you can divide everyone by $g$. And do it until $gcd(a,m) \neq 1$. But after every time you do this operation you eliminate one $a$ from $a^x$.

    $$a^x \equiv b \mod{m}$$
    $$a^{x-1}*a \equiv b \mod{m}$$
    $$a^{x-1}*\frac{a}{g} \equiv \frac{b}{g} \mod{\frac{m}{g}}$$
    
    $$\dots$$
    
    $$a^{x-add}*\frac{a^{add}}{k} \equiv \frac{b}{k} \mod{\frac{m}{k}}$$

    After applying the operation $add$ times you have to solve the problem with new values where $gcd(a,\frac{m}{k}) \neq 1$.

    \lstinputlisting{./solutions/Algebra/discrete_log.cpp}    
        

    \section{Nelsi's anotations}
        \subsection{Kaplansky's Lemma} 
            \subsubsection{Lemma 1}
                Given a set of size $N$, how many subsets of size $K$ that do not contain consecutive elements exist?
                
                Basically, we will rewrite the elements of our set as $+$ and $-$. Such that between each pair of $+$, there is a $-$. Initially, we will choose $K$ plus signs. Then, we will have $x_1$ minus signs before the first plus, $x_2$ between the first and the second, and so on, up to $x_{\scriptscriptstyle K+1}$.
                \begin{align*}
                    x_1, x_{\scriptscriptstyle K+1} &\ge 0\\
                    x_i > 0 \; \forall \; 2 &\le i \le K 
                \end{align*}
        
                We have $N-K$ minus signs to distribute. Since $x_2$ to $x_{\scriptscriptstyle K}$ are positive integers, we will express $x_1$ and $x_{\scriptscriptstyle K+1}$ as positive integers in the form of $x'_1-1$ and $x'_{\scriptscriptstyle K+1}-1$, resulting in the equation:
                
                
                $x'_1 - 1 + x_2 + x_3 + ... + x_{\scriptscriptstyle K-1} + x_{\scriptscriptstyle K} + x'_{\scriptscriptstyle K+1}-1 = N-K$
        
                $x'_1 + x_2 + x_3 + ... + x_{\scriptscriptstyle K-1} + x_{\scriptscriptstyle K} + x'_{\scriptscriptstyle K+1} = N - K + 2$
        
                
                We know the formula for calculating how many integer solutions there are for a positive integer.
        
                $\binom{M-1}{P}$ Assuming $M$ is the desired answer and $P$ is the number of coefficients.
        
                In our problem, it takes the following form: $\binom{N-K+1}{K}$
    
            \subsubsection{Lemma 2}
            Given a circle with numbers from $1$ to $N$, in how many ways can we choose $K$ numbers without selecting neighboring ones?

            We can divide the problem into $2$ cases:

            If N belongs to the set, we need to choose $K-1$ elements from $N-3$ options. Following the previous lemma, our answer is:
            
            $\binom{N-K-1}{K-1}$
            
            
            If N does not belong to the set, we have to choose $K$ from $N-1$ options.Following the previous lemma, our answer is:
            
            $\binom{N-K}{K}$

            
            Summing up the two answers we obtained, we have:  $\binom{N-K-1}{K-1}\binom{N-K}{K}$ $=$ $\frac{N}{N-K}\binom{N-K}{K}$
        \subsection{Divisor Analysis - CSES} 
            \subsubsection{Number of Divisors}
                    Each divisor of the number can be written as $\prod_{i = 1}^N x_i^{\alpha_i}$
                    where $0 \leq \alpha_i \leq k_i$.
                    
                    Since there are $k_i + 1$ choices for $\alpha_i$, the number of divisors is
                    simply $\prod_{i = 1}^N (k_i + 1)$.
                    
                    We can calculate this by iterating through the prime factors in $\mathcal O(N)$
                    time.
                    
            \subsubsection{Sum of divisors}
            
                Let the sum of divisors when only considering the first $i$ prime factors be $S_i$. The answer will be $S_N$.
                
                $$\begin{aligned} S_i &= S_{i - 1} \sum_{j = 0}^{k_i} x_i^j\\ &= S_{i - 1} \cdot \frac{x_i^{k_i + 1} - 1}{x_i - 1}\\ \end{aligned}$$
                
                We can calculate each $S_i$ using fast exponentiation and modular inverses in
            $\mathcal O(N \log(\max(k_i)))$ time.
            
            \subsubsection{Product of divisors}

                Let the product and number of divisors when only considering the first $i$ prime
                factors be $P_i$ and $C_i$ respectively. The answer will be $P_N$.
                
                $$\begin{aligned} P_i &= P_{i - 1}^{k_i + 1} \left(x_i^{k_i(k_i + 1)/2} \right)^{C_{i - 1}} \end{aligned}$$
                
                Again, we can calculate each $P_i$ using fast exponentiation in
                $\mathcal O(N \log(\max(k_i)))$ time, but there's a catch! It might be tempting
                to use $C_{i - 1}$ from your previously-calculated values in part 1 of this
                problem, but those values will yield wrong answers.
                
                This is because $a^b \not \equiv a^{b \bmod p} \pmod{p}$ in general. However, by
                Fermat's little theorem, $a^b \equiv a^{b \bmod (p - 1)} \pmod{p}$ for prime
                $p$, so we can just store $C_i$ modulo $10^9 + 6$ to calculate $P_i$.

        \subsection{Bracket Sequences II - CSES}
            For the "Bracket Sequences II" solution, we will first calculate the factorials and perform the following checks. Initially, we will check if the bracket sequence (BS) becomes irregular at any point; if it does, we return $0$. Additionally, we will verify if our right bracket sequence (RBS) has all $N$ elements; if it does, we return $1$.

            After these base case checks, we will proceed to calculate the Catalan number for this value.
 
            The standard formula for the Catalan number is: \\

            $C_n = \binom{2n}{n} - \binom{2n}{n - 1}$

            This formula means that out of the $2n$ opening parentheses, we will choose $n - 1$ to close. We will make a small modification to it. We will keep track of the number of parentheses that have already been opened and closed and substitute that into the formula. \\

            $C_n = \frac{2n - (open + close)}{(n-open)!(n-close)!} - \frac{2n - (open + close)}{(n-open-1)!(n-close+1)!}$


        
    \section{Multiplicative Functions}
   In number theory, a multiplicative function is an arithmetic function $f(n)$ of a positive integer $n$ with the property that $f(1) = 1$ and

$$\displaystyle f(ab)=f(a)f(b)$$

whenever a and b are coprime.

    Examples of multiplicative functions:
    \begin{itemize}
        \item $gcd(n,k)$: the greatest common divisor of n and k, as a function of n, where k is a fixed integer.
\item {$\displaystyle \varphi (n)$}: Euler's totient function 
{$\displaystyle \varphi$ }, counting the positive integers coprime to (but not bigger than) $n$
\item $\mu (n)$: the Mobius function, the parity (-1 for odd, +1 for even) of the number of prime factors of square-free numbers; 0 if n is not square-free
\item $\sigma k(n)$: the divisor function, which is the sum of the k-th powers of all the positive divisors of n (where k may be any complex number). Special cases we have
\begin{itemize}
    \item $\sigma 0(n) = d(n)$ the number of positive divisors of n,
    \item $\sigma 1(n) = \sigma (n)$, the sum of all the positive divisors of n.
\end{itemize}
    \end{itemize}

    The main ideia of multiplicative functions is to use the linear sieve or just the sieve to calculate them. Normaly it will lead to some principle and u will need to calculate the solution to powers of primes and products of coprimes.
    for example we have for $\phi(x)$ that:
    \begin{itemize}
        \item $\phi(p) = p-1$,
        \item $\phi(p^k) = p^k -p^{k-1}$,
        \item $\phi(a*b) = \phi(a) * \phi(b)$, if $a$ and $b$ are coprime,
        \item $\phi(a*b) = \phi(a) * \phi(b) * \frac{gcd(a,b)}{\phi(gcd(a,b))}$, if $a$ and $b$ arent coprime,
    \end{itemize}

    Then we can calculate phi using sieve using the following code:

    \lstinputlisting{./solutions/Algebra/phi.cpp}

    mantaining the quantity of the smallest prime can be useful to some functions, such as $f(p^k) = k$.

    \lstinputlisting{./solutions/Algebra/fpk.cpp}
    \subsection{Mobius Inversion}

    The Mobius function which is in the list above have a really interesting property:

    lets define a function f as:

    $$
        f(x) = 
      \begin{cases}
        1, & \text{for } x = 1 \\
        0, & \text{otherwise.} 
      \end{cases}
    $$
    
    we have an interesting property which can be used to solve alot of multiplicative function related problems.

    $$
        f(x)  = \sum_{d|x}\mu(d)
    $$

    So lets say we need to calculate the following:
    \begin{align*}
        \sum_{i=1}^{n}& \sum_{j=1}^{n} [gcd(i,j) = 1] \\
        \sum_{i=1}^{n}& \sum_{j=1}^{n} \sum_{d|gcd(i,j)}\mu(d) \\
        \sum_{i=1}^{n}& \sum_{j=1}^{n} \sum_{d = 1}^{n} [d|gcd(i,j)]*\mu(d) \\
        \sum_{i=1}^{n}& \sum_{j=1}^{n} \sum_{d = 1}^{n} [d|i]*[d|j]*\mu(d) \\
       \sum_{d = 1}^{n}& \mu(d) * (\sum_{i=1}^{n}[d|i]) * (\sum_{j=1}^{n}[d|j]) \\
       \sum_{d = 1}^{n}& \mu(d) * (\left\lfloor \frac{n}{d} \right\rfloor) * (\left\lfloor \frac{n}{d} \right\rfloor) 
    \end{align*}

    So we started with a $O(N^2 * log(N))$ solution went to a $O(N^3 * log(N))$ solution, 
    but with the power of magic(known as math in some places) we ended with a $O(N)$ solution.

    Some other possible problems using this:
    
    \begin{align*}
        \sum_{i=1}^{n}& \sum_{j=1}^{n} gcd(i,j) \\
        \sum_{k=1}^{n}& \sum_{i=1}^{n} \sum_{j=1}^{n} [gcd(i,j) = k] \\
        \sum_{k=1}^{n}& k * \sum_{a=1}^{\left\lfloor \frac{n}{k} \right\rfloor} 
        \sum_{b=1}^{\left\lfloor \frac{n}{k} \right\rfloor} [gcd(a,b) = 1] \\
        \sum_{k=1}^{n}& k * \sum_{d = 1}^{\left\lfloor \frac{n}{k} \right\rfloor} \mu(d) * (\left\lfloor \frac{n}{d*k} \right\rfloor) * (\left\lfloor \frac{n}{d*k} \right\rfloor) 
    \end{align*}

    Next:

    \begin{align*}
        \sum_{i=1}^{n}& \sum_{j=1}^{n} lcm(i,j) \\
        \sum_{i=1}^{n}& \sum_{j=1}^{n} \frac{i*j}{gcd(i,j)} \\
        \sum_{k=1}^{n}& \sum_{i=1}^{n} \sum_{j=1}^{n} [gcd(i,j) = k]*\frac{i*j}{k} \\
        \sum_{k=1}^{n}& k * \sum_{a=1}^{\left\lfloor \frac{n}{k} \right\rfloor} 
        \sum_{b=1}^{\left\lfloor \frac{n}{k} \right\rfloor} [gcd(a,b) = 1]*a*b \\
        \sum_{k=1}^{n}& k * \sum_{d = 1}^{\left\lfloor \frac{n}{k} \right\rfloor} \mu(d) * (\sum_{a=1}^{\left\lfloor \frac{n}{k} \right\rfloor}[d|a]*a) * ( 
        \sum_{b=1}^{\left\lfloor \frac{n}{k} \right\rfloor} [d|b]*b) \\
        a =& d*p\text{, }b = d*q\\
        \sum_{k=1}^{n}& k * \sum_{d = 1}^{\left\lfloor \frac{n}{k} \right\rfloor} \mu(d) * (\sum_{p=1}^{\left\lfloor \frac{n}{k*d} \right\rfloor}d*p) * ( 
        \sum_{q=1}^{\left\lfloor \frac{n}{k*d} \right\rfloor} d*q) \\
       \sum_{k=1}^{n}& k * \sum_{d = 1}^{\left\lfloor \frac{n}{k} \right\rfloor} \mu(d) * d^2 * (\sum_{p=1}^{\left\lfloor \frac{n}{k*d} \right\rfloor}p) * ( 
        \sum_{q=1}^{\left\lfloor \frac{n}{k*d} \right\rfloor} q) \\
        \sum_{k=1}^{n}& k * \sum_{d = 1}^{\left\lfloor \frac{n}{k} \right\rfloor} \mu(d) * d^2 * (\sum_{p=1}^{\left\lfloor \frac{n}{k*d} \right\rfloor}p) * ( 
        \sum_{q=1}^{\left\lfloor \frac{n}{k*d} \right\rfloor} q) \\
        \text{the}&\text{ left part can be done with an PA, which will denoted as } PA(l,r)\\
        \sum_{k=1}^{n}& k * \sum_{d = 1}^{\left\lfloor \frac{n}{k} \right\rfloor} \mu(d) * d^2 * (PA(1,\left\lfloor \frac{n}{k*d} \right\rfloor)^2)  \\
        \text{let }& l = k*d\\
         \sum_{k=1}^{n}& k * \sum_{d = 1}^{\left\lfloor \frac{n}{k} \right\rfloor} \mu(d) * d^2 * (PA(1,\left\lfloor \frac{n}{l} \right\rfloor)^2)  \\
         \sum_{l=1}^{n}& (PA(1,\left\lfloor \frac{n}{l} \right\rfloor)^2) * \sum_{d'|l} \mu(d') * d' * l   \\
    \end{align*}

    Now lets analyze a problem with an vector $A$ with max element being $M$ and we need to calculate:


    \begin{align*}
        \sum_{i=1}^{n}& \sum_{j=1}^{n} lcm(A[i],A[j]) \\
        \sum_{d=1}^{M}& \sum_{(a \in A \And d|a)} \sum_{(b \in A \And d|b)} [gcd(\frac{a}{d},\frac{b}{d}) = 1] * \frac{a*b}{d} \\
        \sum_{d=1}^{M}& \sum_{(a \in A \And d|a)} \sum_{(b \in A \And d|b)} \sum_{l=1}^{M} \mu(l) * [l|\frac{a}{d}][l|\frac{b}{d}] * \frac{a*b}{d} \\
        \sum_{d=1}^{M}& \frac{1}{d} * \sum_{l=1}^{M} \mu(l) * ( \sum_{a \in A} \sum_{b \in A} [dl|a][dl|b] * a*b) \\
        \text{let }& t = dl\\
        \sum_{t=1}^{M}& \frac{1}{d} * \sum_{l|t} \frac{l}{t} * \mu(l) * ( \sum_{a \in A} \sum_{b \in A} [t|a][t|b] * a*b) \\
        \sum_{t=1}^{M}&(\sum_{l|t} \frac{l}{t} * \mu(l)) * ( \sum_{a \in A \And t|a} a)^2 
    \end{align*}
    
  