\chapter{Data Structures}
    
    \section{Ordered Set}
    Same as Set, but with two new queries
    \begin{itemize}
        \item s.order\_of\_key$(k)$: Number of items strictly smaller than $k$ .
        \item s.find\_by\_order$(k)$: $k$-th element in a set (counting from zero).
    \end{itemize}
    
    \lstinputlisting{./solutions/DataStructures/oset.cpp}
    
    \bigskip
    \section{Fenwick Tree}
    \subsection{Simples}

    Coloquei só pelo drama do Dayllon.
    Lembra q da pra usar pra maximo e minimo em \textbf{prefixo}. É possível usar em sufixo, é só inverter as operações da bit (soma vai pra query e subtração vai pra update).

    \lstinputlisting{./solutions/DataStructures/fenwicktree.cpp}

    Versão que o Sabino gosta

    \lstinputlisting{./solutions/DataStructures/fenwicktree2.cpp}
    
    \subsection{2D}
    You are given an n×n grid representing the map of a forest. Each square is either empty or has a tree. Your task is to process q queries of the following types:
    Change the state (empty/tree) of a square.
    How many trees are inside a rectangle in the forest?

    You can use the magic of Inclusion and Exclusion to solve this one.
    \lstinputlisting{./solutions/DataStructures/ForestQueriesII.cpp}
    \subsection{Fenwick Lifting}
    Binary Search on a BIT in $O(log(N))$
    
    \lstinputlisting{./solutions/DataStructures/fenwick_lifting.cpp}
    \section{Sparse Table}
    
    \subsection{Simple Sparse table}
    idempotent queries in $O(1)$, others $O(\log (n))$;
    \lstinputlisting{./solutions/DataStructures/sparsetable.cpp}
    Versão que o Sabino gosta
    \lstinputlisting{./solutions/DataStructures/sparsetable2.cpp}
    % \subsection{2D Sparse table}
    % idempotent queries in $O(1)$;
    % CODA ESSE NEGOCIO, O CODIGO QUE TA LA USA $\ log2$ GUSTAVO, MDS DO CEU
    % \lstinputlisting{./solutions/DataStructures/2Dsparsetable.cpp}
    \section{SQRT Decomposition}
        \subsection{MO}
        Nada de novo sob o sol, lembre q na aba de DSU persistente tem uma aplicação bem hard.
         \lstinputlisting{./solutions/DataStructures/mo.cpp}
         Versão 2, só existe aqui pq o Sabino gosta desse
         \lstinputlisting{./solutions/DataStructures/mo2.cpp}
        \subsection{MO com queries}
         \lstinputlisting{./solutions/DataStructures/moupdate.cpp}
         Um ponto importante é que se nós fizermos aquele esquema de decompor a query de path em arvore em uma query no range do euler tour, também podemos usar mo.
        \subsection{Blocking}
        We partition the array into blocks of size $\texttt{block\_size}=\lceil \sqrt{N} \rceil$. Each block stores the sum of elements within it, and allows for the
        creation of corresponding update and query operations.
        
        Update Queries: $\mathcal{O}(1)$
        
        To update an element at location $x$, first find the corresponding block using
        the formula $\frac{x}{\texttt{block\_size}}$.
        
        Then, apply the corresponding difference between the element currently stored at
        $x$ and the element we want to change it to.
        
        Sum Queries: $\mathcal{O}(\sqrt{N})$
        
        To perform a sum query from $[0\ldots r]$, calculate
        
        $$\sum_{i = 0}^{R-1} \texttt{blocks}[i] + \sum_{R \cdot \texttt{block\_size}}^r \texttt{nums}[i]$$
        
        where $\texttt{blocks}[i]$ represents the total sum of the $i$-th block, the
        $i$-th block represents the sum of the elements from the range $[i\cdot \texttt{block\_size},(i + 1)\cdot \texttt{block\_size})$, and $R=\left\lceil \frac{r}{\texttt{block\_size}} \right\rceil$.
        
        Finally, $\sum_{i=l}^{r} \texttt{nums}[i]$ is the difference between the two
        sums $\sum_{i=0}^{r}\texttt{nums}[i]$ and $\sum_{i=0}^{l-1}\texttt{nums}[i]$,
        which each are calculated in $\mathcal{O}(\sqrt N)$.

        \lstinputlisting{./solutions/DataStructures/blocking.cpp}
        \subsection{Blocking com cyclic shift}
        Legendary Huron is a fan of beautiful fences. He defines the beauty of a sequence (l,r) is $max_{l \leq i \leq j \leq r}(a[j]-a[i])$
        Queries
        
        Given three integers l, r, k, he will repaint the planks in the range [l,r]. With a[l] = k, a[l+1] = k+1 ...;
        Given two integers l and r, returns the beauty.
        \lstinputlisting{./solutions/DataStructures/sqrtdecomp.cpp}
        \subsection{Batching}
        Maintain a "buffer" of the latest updates (up to $\sqrt N$). The answer for each
        sum query can be calculated with prefix sums and by examining each update within
        the buffer. When the buffer gets too large ($\ge \sqrt N$), clear it and
        recalculate prefix sums.
        
        \lstinputlisting{./solutions/DataStructures/batching.cpp}
        
    \section{Treap}
    Basicamente é uma estrutura que dá pra fazer um monte de coisa com "pouco"  código. A operação reverse é a + roubada.
    
    \lstinputlisting{./solutions/DataStructures/Treap.cpp}

    \section{DSU}
    There are multiples implementations here, with different focuses.

    \subsection{Persistent DSU}
    First one: Just an DSU such that u can ask who was my parent in time $t$.
    \lstinputlisting{./solutions/DataStructures/dsu_rollback_1.cpp}

    
    \subsection{DSU with rollback}
    Second: an DSU such that u roll back last operation.
    \lstinputlisting{./solutions/DataStructures/dsu_rollback_2.cpp}

    Now, some crazy applications.
    \subsection{Connected Components With Segments}
    
    First one, given some set of edges, answer multiples queries. For each query answer how many connected components are there, if u use the edges in range (l,r).

    The solution does an approach with mo and dsu with rollback. The main ideia is:

    First solve for those such that $(r-l+1) < \sqrt(N)$. Just brute with DSU.
    After it, lets split the solution in blocks of $l$. 
    For each block u sort the queries by r. Lets say u are in first block and every $l$ lies in range $1--\sqrt{N}$.
    The solution does the following steps:
    \begin{itemize}
        \item Add every edge in range $\sqrt{N}+1-r$.
        \item Add every edge in range $l-\sqrt{N}$.
        \item get answer for this segment.
        \item rollback the edges in range $l-\sqrt{N}$.
        \item go to the next segment.
    \end{itemize}

    The first step can be slow, but for each block the amortized time will be $N$. The other steps, take $\sqrt{N}$ time.

    \lstinputlisting{./solutions/DataStructures/ConnectedComponentsWithSegments.cpp}

    \subsection{Dynamic Connectivity Offline}

    Last application. solve those queries:
    \begin{itemize}
        \item add an edge $u - v$
        \item remove an edge $u - v$
        \item answer the number off connected components.
    \end{itemize}

    The main idea is to define an id to each query of type $3$ and define ranges off "activation" suppose u add an edge $u - v$ and after it u remove. This will create a range $l - r$ which represents which queries off type $3$ this edge is active.

    After it we will do a "segment tree" alike solution.

    The main idea is to keep the segments which are relevant to this part of the queries and just add the edge when it cover all the ids in this segment. after it u always roll back the edges added in this level. it is easy to see (maybe it is not) that each segment will have a complexity close to a query in a segment tree.

    \lstinputlisting{./solutions/DataStructures/DynamicConnectivityOffline.cpp}


    \subsection{DSU with Small to Large}

    When you have DSU and besides the representative thing, you may have other information about the groups, useful to solve queries about unions.

    \lstinputlisting{./solutions/DataStructures/DSU_SMOL.cpp}
    
    

    
    
    
    