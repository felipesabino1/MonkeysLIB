\chapter{Miscellaneous}
    \section{Game Theory}
        Pontos importantes:
        \begin{itemize}
            \item \textit{Teoria do Espelhamento}: Se o seu oponente pode espelhar todas as suas ações, este é um estado de derrota.
            \item Dois nim games são combinados usando o XOR.
            \item Podem existir ciclos modulares, vale brutar.
            \item Em algum momento pode se tornar sempre win, ou sempre loss.
            \item Se a gente transformar o problema num nim, podemos usar o teorema de grundy, que basicamente é achar os casos derrota + fazer o mex.
        \end{itemize}
        
        Example 1:

        There is a heap of $n$ coins and two players who move alternately. On each move, a player chooses a heap and divides into two nonempty heaps that have a different number of coins. The player who makes the last move wins the game.

        Here for big N the answer is always first.

        
    \lstinputlisting{./solutions/Miscellaneous/GrundyGame.cpp}

        Example 2:

        There is a staircase consisting of $n$ stairs, numbered $1,2,\ldots,n$. Initially, each stair has some number of balls.

        There are two players who move alternately. On each move, a player chooses a stair $k$ where $k\neq1$ and it has at least one ball. Then, the player moves any number of balls from stair $k$ to stair $k-1$. The player who moves last wins the game.

        Here, every odd position is losing, u can see them as "dark holes", using the "Teoria do Espelhamento", so it's just N/2 Nim games.

        
    \lstinputlisting{./solutions/Miscellaneous/StairGame.cpp}

    \subsection{Nim Multiplication}
        A 2D NIM Game int the form of NXM can be reduced
        to the Nim Multiplication of 2 1D NIM GAMES
        in the form of NimMult(Grundy(N),Grundy(M))
        
        \lstinputlisting{./solutions/Miscellaneous/NimMult.cpp}
\section{Binary Search}
    \subsection{Parallel Binary Search}
    (Original Problem: Atcoder Grand Contest 2 D)
     \lstinputlisting{./solutions/Miscellaneous/agc002_d.cpp}
% \section{Next\_Permutation - C++}
%     \lstinputlisting{./solutions/Miscellaneous/next_permutation.cpp}
\section{Big Num}
    Implementação de Big Number em C++.
        \lstinputlisting{./solutions/Miscellaneous/bignum.cpp}
    \subsection{D\&C for Multiplyng two big numbers}
        \lstinputlisting{./solutions/Miscellaneous/karatsuba.cpp}

\section{Bitsets}
Some $O(N^2)$ solutions can be optimized using bitsets. Mainly knapsack problems.
An important point is to use the "popcount" optimization  in target of pragma.

how to initialize:

\begin{lstlisting}
bitset<size> variable_name;
bitset<size> variable_name(DECIMAL_NUMBER);
bitset<size> variable_name("BINARY_STRING")
\end{lstlisting}

Every position can be accessed as a vector, also size must be constant. We get an constant optimization of 32. every bit operation used in integers can be used in bitsets.

Some functions in bitsets:
\begin{center}
\begin{tabular}{ c | l }
  set()  & Set the bit value at all bitset to 1.  \\ 
 reset() & Set the bit value at all bitset to 0.  \\  
 flip() & Flip the bit value at all bitset.\\
 set(idx) & Set the bit value at the given idx to 1.  \\ 
 reset(idx) & Set the bit value at a given idx to 0.  \\  
 flip(idx) & Flip the bit value at the given idx.\\
 count() & Count the quantity of bits set.\\  
 any() & Checks if any bit is set.\\
 all() & Check if all bit is set.\\
 none() & Checks if none bit is set.\\
 size() & Returns the size of the bitset.\\
 to\_string() & Converts bitset to std::string.\\
 to\_ulong() & Converts bitset to unsigned long.\\
 to\_ullong() & Converts bitset to unsigned long long.\\
 \_Find\_first() & return index of first bit set. \\
 \_Find\_next(idx) & return index of first bit set after idx.
\end{tabular}
\end{center}

\section{Built in functions}

Some Built in functions in GCC(remember that if u are using ll, need to add ll at the end):
\begin{center}
\begin{tabular}{ c | l }
  \_\_builtin\_popcount(x)  & Counts the quantity of one’s(set bits) in an integer.  \\ 
 \_\_builtin\_parity(x) & Checks the Parity of a integer.\\
 
 &Returns true(1) if odd parity(odd quantity of set bits)\\
 
 & Returns false(0) for even parity(even quantity of set bits).  \\  
 \_\_builtin\_clz(x) &  Counts the leading quantity of zeros of the integer.\\
 \_\_builtin\_ctz(x) & Counts the trailing quantity of zeros of the integer.  \\ 
 \_\_builtin\_popcountll(x)  & Counts the quantity of one’s(set bits) in an long long.  \\ 
 \_\_builtin\_parityll(x) & Checks the Parity of a long long.\\
 
 &Returns true(1) if odd parity(odd quantity of set bits)\\
 
 & Returns false(0) for even parity(even quantity of set bits).  \\   
 \_\_builtin\_clzll(x) &  Counts the leading quantity of zeros of the long long.\\
 \_\_builtin\_ctzll(x) & Counts the trailing quantity of zeros of the long long.  \\ 
\end{tabular}
\end{center}
clz can be used to find the first set bit, since u can use:
\begin{lstlisting}
int x;
ll xl;
int fsi =  31 - __builtin_clz(x);
int fsll = 63 - __builtin_clzll(xl);
\end{lstlisting}
\section{Priority Queue and Set Comparators}

Set Comparators
Operator Overloading:
\begin{lstlisting}
#include <bits/stdc++.h>
using namespace std;

struct Edge {
	int a, b, w;
	bool operator<(const Edge &y) const { return w < y.w; }
};

int main() {
	int M = 4;
	set<Edge> v;
	for (int i = 0; i < M; ++i) {
		int a, b, w;
		cin >> a >> b >> w;
		v.insert({a, b, w});
	}
	for (Edge e : v) cout << e.a << " " << e.b << " " << e.w << "\n";
}
\end{lstlisting}
Functors:

\begin{lstlisting}
#include <bits/stdc++.h>
using namespace std;

struct Edge {
	int a, b, w;
};

struct cmp {
	bool operator()(const Edge &x, const Edge &y) const { return x.w < y.w; }
};

int main() {
	int M = 4;
	set<Edge, cmp> v;
	for (int i = 0; i < M; ++i) {
		int a, b, w;
		cin >> a >> b >> w;
		v.insert({a, b, w});
	}
	for (Edge e : v) cout << e.a << " " << e.b << " " << e.w << "\n";
}
\end{lstlisting}

Functors can also be used in priority queues, but needs vector as a container:

\begin{lstlisting}
priority_queue<int, vector<int>, cmp> c; 
\end{lstlisting}
Built in Functors:
\begin{lstlisting}
set<int, greater<int>> a; //max set
map<int, string, greater<int>> b; // max map
priority_queue<int, vector<int>, greater<int>> c; // min heap
\end{lstlisting}
\section{Lambda Expressions}
Basic Lambda Syntax
    A lambda expression consists of the following:
\begin{center}
    [capture list] (parameter list) {function body}
\end{center}
The capture list and parameter list can be empty, so the following is a valid lambda:
\begin{lstlisting}
[](){cout << "Hello, world!" << endl;}
auto func1 = [](int i) {cout << ":" << i << ":";};
func1(x);
\end{lstlisting}
Using \& in capture list it will have access to the scope.

Recursive lambda using function:
\begin{lstlisting}
int main() {
	function<int(int, int)> gcd = [&](int a, int b) {
		return b == 0 ? a : gcd(b, a % b);
	};
	cout << gcd(20, 30) << '\n';  // outputs 10
}
\end{lstlisting}
    
\section{Fast input output}
        \lstinputlisting{./solutions/Miscellaneous/IOPT.cpp}
\section{Trick for faster Unordered Map}
        \lstinputlisting{./solutions/Miscellaneous/umap.cpp}
\section{Faster Hash Table}
    \lstinputlisting{./solutions/Miscellaneous/hashtablerobada.cpp}
\section{StringStream}
    A way to parse strings in c++.
    \lstinputlisting{./solutions/Miscellaneous/stringstream.cpp}
\section{Karmarkar-Karp}
    Heuristic algorithm to divide a set in two others set, in the way that the difference between the sum of this two new sets will be minimal.
    The error of this heuristc algorithm is $\mathcal{O}(n^{-lg(n)})=\mathcal{O}(\frac{1}{n^{lg(n)}})$
    \lstinputlisting{./solutions/Miscellaneous/karmakarp.cpp}