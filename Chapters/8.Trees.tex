\chapter{Trees}
    \section{Diameter}
    Finding the diameter of the tree. Just do two DFS's, where in the first one u find the longest guy u can reach, and then find the other guy starting from the one you found before.
    \lstinputlisting{./solutions/Tree/diameter.cpp}
    \section{LCA}
    A way to find the lowest common ancestor of two vertex $(a,b)$ in a tree. Needs a Sparse Table.
    \lstinputlisting{./solutions/Tree/lca.cpp}
    Version that Sabino likes
    \lstinputlisting{./solutions/Tree/lca2.cpp}

    
    \section{Euler tour tree}

    No code here, use lca code. remember u can calc paths using the ett. There are two types.
    
    Incrementing $out$, and not incrementing $out$. We can use the first one as a kind of bit with $f(x)$ in $in[x]$ and $f(x)^{-1}$ in $out[x]$
    
    \section{Small to Large}

    Not much to say, the code is always different, so lets just keep the USACO text:

    Naive Solution

    Suppose that we want merge two sets $a$ and $b$ of sizes $n$ and $m$,
    respectively. One possibility is the following:
    
    \lstinline{for (int x : b) a.insert(x);}
    
    which runs in $\mathcal{O}(m\log (n+m))$ time, yielding a runtime of
    $\mathcal{O}(N^2\log N)$ in the worst case. If we instead maintain $a$ and $b$
    as sorted vectors, we can merge them in $\mathcal{O}(n+m)$ time, but
    $\mathcal{O}(N^2)$ is also too slow.
    
    Better Solution
    
    With just one additional line of code, we can significantly speed this up.
    
    \lstinline{if (a.size() < b.size()) swap(a, b);for (int x : b) a.insert(x);}
    
    Note that swap exchanges two
    sets in $\mathcal{O}(1)$ time. Thus, merging a smaller set of size $m$ into the
    larger one of size $n$ takes $\mathcal{O}(m\log n)$ time.
    
    Claim: The solution runs in $\mathcal{O}(N\log^2N)$ time.
    
    Proof: When merging two sets, you move from the smaller set to the larger
    set. If the size of the smaller set is $X$, then the size of the resulting set
    is at least $2X$. Thus, an element that has been moved $Y$ times will be in a
    set of size at least $2^Y$, and since the maximum size of a set is $N$ (the
    root), each element will be moved at most $\mathcal{O}(\log N$) times.

    Generalizing

    We can also merge other standard library data structures such as std::map or
    std:unordered\_map in the same way. However,
    std::swap does not always
    run in $\mathcal{O}(1)$ time. For example, swapping
    std::arrays takes time
    linear in the sum of the sizes of the arrays, and the same goes for
    GCC policy-based data structures such
    as \_\_gnu\_pbds::tree or \_\_gnu\_pbds::gp\_hash\_table.
    
    To swap two policy-based data structures a and b in $\mathcal{O}(1)$ time,
    use a.swap(b) instead. Note that for standard library data structures,
    swap(a,b) is equivalent to a.swap(b).
    
    \section{HLD}
    You are given a tree consisting of n nodes. The nodes are numbered 1,2,…,n. Each node has a value.

    Your task is to process following types of queries:
    \begin{itemize}
        \item Change the value of node s to x
        \item Find the maximum value on the path between nodes a and b.
    \end{itemize}

    \lstinputlisting{./solutions/Tree/PathQueriesII.cpp}
    \subsection{Can you answer these queries VII}
        Given are a tree with $n$ vertices, each vertice has a weight $x_i$, and two type of operations. Operation 1 asks for the maximum contiguous sum between two vertices $a$ and $b$. Operation 2 update the values of vertices between $a$ and $b$ to $c$. We have to process $Q$ operations.
        
        The problem can be solved with HLD in $O(Q(log(n))^2)$. Original Problem (SPOJ - GSS7).
        
        \lstinputlisting{./solutions/Tree/HLD.cpp}
        
        \subsection{Can you answer these queries VI}
        Given a tree with $n$ vertices, each vertice has a color (0 or 1), and two type of operations. Operation 0 asks for how many vertices are connected to a vertice $u$, when one vertice $v$ is connected to $u$ only if the path between $u$ and $v$ just contain vertices with same color of $u$ inclusive. Operation 1 is to change the color of vertice $u$ (0 to 1 and 1 to 0).
        
        The problem can be solved wih HLD in $O(Q(log(n))^2)$. Original Problem (SPOJ - Can You Answer These Queries VI).
        
        
        \lstinputlisting{./solutions/Tree/CanYouAnswerTheseQueriesVI.cpp}

         \subsection{HLD with subtree queries}
        Slower version with subtree queries, uses lazy segtree and  template
        
        
        \lstinputlisting{./solutions/Tree/HLD_subtree.cpp}
    \section{Centroid Decomposition}


    Always about path problems.
    
    There are two types of centroid solving. Do some computation during centroid and use it.
     Or keep some data for each node and make updates in $O(log(N))$. 

    \lstinputlisting{./solutions/Tree/centroid.cpp}

    Version that Sabino likes

    \lstinputlisting{./solutions/Tree/centroid2.cpp}

    Example 1:
    Given a tree of $n$ nodes, your task is to count the number of distinct paths that consist of exactly $k$ edges.

    \lstinputlisting{./solutions/Tree/FixedLengthPathsI.cpp}
    
    Example 2: 
    Xenia has a tree consisting of $n$ nodes. We will consider the tree nodes indexed from $1$ to $n$. We will also consider the first node to be initially painted red, and the other nodes — to be painted blue.
    
    execute queries of two types:
    \begin{itemize}
        \item paint a specified blue node in red;
        \item calculate which red node is the closest to the given one and print the shortest distance to the closest red node.
    \end{itemize}

    \lstinputlisting{./solutions/Tree/xenia&tree.cpp}
    
    
    \section{Tree Isomorphism}

    In this code, we generate an id for each tree.
    It is also possible to do with subtrees.

    \lstinputlisting{./solutions/Tree/treehash.cpp}

    \section{Virtual Tree}
    A different way to approach trees problems. Main idea here is to create a tree with the "relevant nodes". Normally related to colors and problems which we solve looking for each color at a time.

    
    \lstinputlisting{./solutions/Tree/virtual_tree.cpp}
    
    
    \section{Link Cut Tree}
    A link cut tree is a data structure that uses splay trees to represent a forest of rooted trees and can perform the following operations with an amortized upper bound time complexity of $\mathcal{O}(\log N)$:
    \begin{itemize}
        \item Linking a tree with a node by making the root of the tree a child of any node of another tree
        \item Deleting the edge between a node and its parent, detaching the node's subtree to make a new tree
        \item Find the root of the tree that a node belongs to
    \end{itemize}

    These operations all use the $\texttt{access}(v)$ subroutine,
which creates a preferred path from the root of the represented tree to vertex $v$, making a corresponding auxiliary splay tree with $v$ as the root.

    \lstinputlisting{./solutions/Tree/linkcut.cpp}
    
    
    